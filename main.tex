%%
%% Copyright (c) 2018 The Authors.  All Rights Reserved.
%%
%% Weitian LI, et al.
%% School of Physics and Astronomy, Shanghai Jiao Tong University,
%% Shanghai, China.
%%
%% 2018-08-23
%%

\documentclass[letters,a4paper,fleqn,usenatbib]{mnras}
% Available options:
% - letters : for papers in the journal's Letters section (<=5 pages)
% - onecolumn : single column
% - doublespacing : double line spacing (do NOT submit in this format)
% - usenatbib : (always use this) use `natbib' package for citations
% - usegraphicx : includes the `graphicx' package
% - useAMS : support 3 upright Greek characters
% - usedcolumn : use `dcolumn' package for table column alignment

\usepackage{newtxtext,newtxmath}
\usepackage[T1]{fontenc}
\usepackage{ae,aecompl}

%
% Custom packages
%
\usepackage{graphicx}
\usepackage{amsmath}
\usepackage{amssymb}
\usepackage{siunitx}  % typeset units; from `texlive-science'

\graphicspath{{./}{figures/}}  % NOTE: the trailing '/' matters

\sisetup{
  range-phrase=\text{--},
  range-units=single,
  product-units=power,
  list-separator={, },
  list-final-separator={, and },
}
\DeclareSIUnit\MHz{\mega\hertz}

\def\sectionautorefname{Section}
\def\subsectionautorefname{Section}
\def\figureautorefname{Figure}
\def\tableautorefname{Table}

%
% Custom commands
%
\newcommand{\Hi}{\ion{H}{i}}


%%======================================================================
%% Title page
%%

%      ............................................. (<=45 chars)
\title[EoR Separation with CDAE]{%
  EoR Signal Separation Using Convolutional Denoising Autoencoder
}

% If you need two or more lines of authors, add an extra line using \newauthor
\author[Li~et~al.]{%
Weitian Li,$^{1}$\thanks{E-mail:
  \href{mailto:liweitianux@sjtu.edu.cn}{liweitianux@sjtu.edu.cn} (WL);
  \href{mailto:hgxu@sjtu.edu.cn}{hgxu@sjtu.edu.cn} (HX)}
Haiguang Xu,$^{1,2,3}$\footnotemark[1]
Zhixian Ma,$^{4}$
Ruimin Zhu,$^{5}$
Dan Hu,$^{1}$
Zhenghao Zhu,$^{1}$
\newauthor
Chenxi Shan,$^{1}$
Jie Zhu$^{4}$
and
Xiang-Ping Wu$^{6}$
\\
% List of institutions
$^{1}${School of Physics and Astronomy,
  Shanghai Jiao Tong University,
  800 Dongchuan Road, Shanghai 200240, China} \\
$^{2}${Tsung-Dao Lee Institute,
  Shanghai Jiao Tong University,
  800 Dongchuan Road, Shanghai 200240, China} \\
$^{3}${IFSA Collaborative Innovation Center,
  Shanghai Jiao Tong University,
  800 Dongchuan Road, Shanghai 200240, China} \\
$^{4}${Department of Electronic Engineering,
  Shanghai Jiao Tong University,
  800 Dongchuan Road, Shanghai 200240, China} \\
$^{5}${Department of Statistics,
  Northwestern University,
  2006 Sheridan Road, Evanston, IL 60208, US} \\
$^{6}${National Astronomical Observatories,
  Chinese Academy of Sciences,
  20A Datun Road, Beijing 100012, China}
}

% These dates will be filled out by the publisher
\date{Accepted XXX. Received YYY; in original form ZZZ}

% Enter the current year, for the copyright statements etc.
\pubyear{2018}

% Don't change these lines
\begin{document}
\label{firstpage}
\pagerange{\pageref{firstpage}--\pageref{lastpage}}
\maketitle

%
% Abstract
% (<=200 words for Letters)
%
\begin{abstract}
The \SI{21}{\cm} emission from the neutral hydrogen (\ion{H}{i})
is regarded as the decisive probe to explore the epoch of reionization.
\end{abstract}

% Select between one and six entries from the list of approved keywords.
% Don't make up new ones.
% https://academic.oup.com/DocumentLibrary/mnras/keywords.pdf
\begin{keywords}
methods: data analysis --
techniques: interferometric --
dark ages, reionization, first stars --
radio continuum: general
\end{keywords}


%%======================================================================
%% Paper body
%%

\section{Introduction}
\label{sec:intro}

The epoch of reionization (EoR; $z \sim \numrange{6}{15}$) is a period
of the Universe that is still poorly understood.
The \SI{21}{\cm} line emission of the neutral hydrogen (\Hi), which is
redshifted to frequencies below \SI{200}{\MHz}, is regarded as the
decisive probe to directly explore the EoR
\citep[see][for reviews]{furlanetto2006rev,furlanetto2016rev}.
In order to probe the EoR, several low-frequency radio interferometers
are built or to be built to target the \SI{21}{\cm} signal, among which
there are 21CMA \citep{zheng2016}, MWA \citep{tingay2013},
LOFAR \citep{vanHaarlem2013}, PAPER \citep{parsons2010},
HERA \citep{deboer2017}, and SKA \citep{koopmans2015rev}.
The observational challenges, however, are immense due to a variety of
complicated instrumental effects, ionospheric distortions, radio frequency
interference, and the strong astronomical foreground contamination that
overwhelms the EoR signal by about \numrange{4}{5} orders of magnitudes
\citep[see][for a review]{morales2010rev}.

Fortunately, the severe foreground contamination is expected to be
spectrally smooth, while the EoR signal fluctuates rapidly along the
frequency dimension.
This important difference is the key characteristic exploited by many
foreground removal methods in order to uncover the faint EoR signal in
the presence of overwhelming foreground contamination, including the
parametric fitting methods \citep[e.g.,][]{wang2006,liu2009fgrm,wang2013}
and non-parametric methods \citep[e.g.,][]{harker2009,chapman2013,mertens2018}.

However, the frequency-dependent beam effects can damage the smoothness
of the foreground spectrum \citep{liu2009ps}.
The synthesised beam, which is also called the point spread function (PSF)
and is the Fourier transform of the interferometer layout, has jagged
side-lobes extending far beyond the central peak due to the incomplete
$uv$ coverage.
In addition, both the shape and response of the beams vary with
observing frequencies.
Therefore, any unresolved or mis-subtracted foreground sources leave
rapidly oscillating residuals along the frequency dimension, which may
seriously compromise the separation of the EoR signal.

Given the complexity of beam profiles, it is extremely difficult to
craft a parametric model for existing foreground removal methods to
overcome the intricate beam effects.
The data-driven modelling method is thus more feasible and appealing.
In recent years, the deep learning algorithms have seen prosperous
developments and brought breakthroughs into a variety of fields, such
as image classification, speech recognition, and object detection
\citep{lecun2015}.
Among all kinds of neural network architectures, the autoencoder
aims to learn robust features in the data \citep{vincent2008}
and has been widely applied to
dimensionality reduction \citep{hinton2006},
image inpainting and denoising \citep{suganuma2018},
speech separation \citep{grais2017}, and so on.

In this paper, we propose a novel method to separate the EoR signal by
making use of the convolutional denoising autoencoder (CDAE), a common
variant of autoencoders that is designed to learn noise robust features.
We introduce the CDAE and elaborate the proposed method in
\autoref{sec:method}.
The method is then applied to the simulated SKA image cubes to
evaluate its performance, as demonstrated in \autoref{sec:experiments},
where we also carry out a comparison of performance between our
deep-learning-based method and traditional polynomial fitting method.
Finally, we conclude our work in \autoref{sec:conclusions}.


%%======================================================================
\section{Methodology}
\label{sec:method}

We first introduce the CDAE, and then describe the network architecture
that we propose...


%%----------------------------------------------------------------------
\subsection{Convolutional Denoising Autoencoder}
\label{sec:cdae}

The autoencoder (AE) is ...

The convolutional AE (CAE) is ...

The convolutional denoising AE (CDAE) is ...


%%----------------------------------------------------------------------
\subsection{Proposed Network Architecture}
\label{sec:architecture}

The network that we propose is a one-dimensional (1D) CDAE ...


%%----------------------------------------------------------------------
\subsection{Data Preprocessing and Training}
\label{sec:data}

To better train the CDAE, the input data should be appropriately
preprocessed.


%%----------------------------------------------------------------------
\subsection{Evaluation Index}
\label{sec:index}

To evaluate the EoR signal separation performance, an index must be defined...


%%======================================================================
\section{Experiments}
\label{sec:expriments}

We simulate a data set to train the CDAE and evaluate the performance.
We also carry out comparison of performance between our proposed method
and the traditional polynomial fitting method.


%%----------------------------------------------------------------------
\subsection{Data Simulation}
\label{sec:simulation}

We take the \SIrange{154}{162}{\MHz} frequency band as an example.
We simulate the sky images of the EoR signal, Galactic diffuse emission,
extragalactic point sources, and radio haloes.
The SKA1-Low layout configuration is employed to simulate the SKA1
`observed' images.
In this way, we take into account the complicated instrumental effects
of radio interferometers.
See \citealt{li2018} for more details.


%%----------------------------------------------------------------------
\subsection{Results}
\label{sec:results}

The training results ...
The EoR separation performance ...


%%----------------------------------------------------------------------
\subsection{Comparison}
\label{sec:comparison}

Compare to the traditional polynomial fitting method ...


%%======================================================================
\section{Conclusions}
\label{sec:conclusions}

We have proposed a CDAE to separate the faint EoR signal along the
frequency dimension and achieved excellent results...


%%======================================================================
\section*{Acknowledgements}

This work is supported by
the Ministry of Science and Technology of China
(grant No. 2018YFA0404601),
the National Natural Science Foundation of China
(grant Nos. 11433002, 11621303, 61371147),
and the National Key Research and Discovery Plan
(grant No. 2017YFF0210903).


%%======================================================================
%% References

\bibliographystyle{mnras}
\bibliography{references}


%%======================================================================
%% Appendix

% \appendix


%%======================================================================
% Don't change these lines
\bsp	% typesetting comment
\label{lastpage}
\end{document}

%% EOF

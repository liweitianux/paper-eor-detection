%%
%% Copyright (c) 2018 The Authors.  All Rights Reserved.
%%
%% Weitian LI, et al.
%% School of Physics and Astronomy, Shanghai Jiao Tong University,
%% Shanghai, China.
%%
%% 2018-08-23
%%

\documentclass[letters,a4paper,fleqn,usenatbib]{mnras}
% Available options:
% - letters : for papers in the journal's Letters section (<=5 pages)
% - onecolumn : single column
% - doublespacing : double line spacing (do NOT submit in this format)
% - usenatbib : (always use this) use `natbib' package for citations
% - usegraphicx : includes the `graphicx' package
% - useAMS : support 3 upright Greek characters
% - usedcolumn : use `dcolumn' package for table column alignment

\usepackage{newtxtext,newtxmath}
\usepackage[T1]{fontenc}
\usepackage{ae,aecompl}

%
% Custom packages
%
\usepackage{graphicx}
\usepackage{amsmath}
\usepackage{amssymb}
\usepackage{siunitx}  % typeset units; from `texlive-science'

\graphicspath{{./}{figures/}}  % NOTE: the trailing '/' matters

\sisetup{
  range-phrase=\text{--},
  range-units=single,
  product-units=power,
  list-separator={, },
  list-final-separator={, and },
}
\DeclareSIUnit\MHz{\mega\hertz}

\def\sectionautorefname{Section}
\def\subsectionautorefname{Section}
\def\figureautorefname{Figure}
\def\tableautorefname{Table}

%
% Custom commands
%
\newcommand{\Hi}{\ion{H}{i}}


%%======================================================================
%% Title page
%%

%      ............................................. (<=45 chars)
\title[EoR Separation with CDAE]{%
  EoR Signal Separation Using Convolutional Denoising Autoencoder
}

% If you need two or more lines of authors, add an extra line using \newauthor
\author[Li~et~al.]{%
Weitian Li,$^{1}$\thanks{E-mail:
  \href{mailto:liweitianux@sjtu.edu.cn}{liweitianux@sjtu.edu.cn} (WL);
  \href{mailto:hgxu@sjtu.edu.cn}{hgxu@sjtu.edu.cn} (HX)}
Haiguang Xu,$^{1,2,3}$\footnotemark[1]
Zhixian Ma,$^{4}$
Ruimin Zhu,$^{5}$
Dan Hu,$^{1}$
Zhenghao Zhu,$^{1}$
\newauthor
Chenxi Shan,$^{1}$
Jie Zhu$^{4}$
and
Xiang-Ping Wu$^{6}$
\\
% List of institutions
$^{1}${School of Physics and Astronomy,
  Shanghai Jiao Tong University,
  800 Dongchuan Road, Shanghai 200240, China} \\
$^{2}${Tsung-Dao Lee Institute,
  Shanghai Jiao Tong University,
  800 Dongchuan Road, Shanghai 200240, China} \\
$^{3}${IFSA Collaborative Innovation Center,
  Shanghai Jiao Tong University,
  800 Dongchuan Road, Shanghai 200240, China} \\
$^{4}${Department of Electronic Engineering,
  Shanghai Jiao Tong University,
  800 Dongchuan Road, Shanghai 200240, China} \\
$^{5}${Department of Statistics,
  Northwestern University,
  2006 Sheridan Road, Evanston, IL 60208, US} \\
$^{6}${National Astronomical Observatories,
  Chinese Academy of Sciences,
  20A Datun Road, Beijing 100012, China}
}

% These dates will be filled out by the publisher
\date{Accepted XXX. Received YYY; in original form ZZZ}

% Enter the current year, for the copyright statements etc.
\pubyear{2018}

% Don't change these lines
\begin{document}
\label{firstpage}
\pagerange{\pageref{firstpage}--\pageref{lastpage}}
\maketitle

%
% Abstract
% (<=200 words for Letters)
%
\begin{abstract}
The \SI{21}{\cm} emission from the neutral hydrogen (\ion{H}{i})
is regarded as the decisive probe to explore the epoch of reionization.
\end{abstract}

% Select between one and six entries from the list of approved keywords.
% Don't make up new ones.
% https://academic.oup.com/DocumentLibrary/mnras/keywords.pdf
\begin{keywords}
methods: data analysis --
techniques: interferometric --
dark ages, reionization, first stars --
radio continuum: general
\end{keywords}


%%======================================================================
%% Paper body
%%

\section{Introduction}
\label{sec:intro}

In this paper, we propose a CDAE to...

This paper is organised as follows.


%%======================================================================
\section{Methodology}
\label{sec:method}

We first introduce the CDAE, and then describe the network architecture
that we propose...


%%----------------------------------------------------------------------
\subsection{Convolutional Denoising Autoencoder}
\label{sec:cdae}

The autoencoder (AE) is ...

The convolutional AE (CAE) is ...

The convolutional denoising AE (CDAE) is ...


%%----------------------------------------------------------------------
\subsection{Proposed Network Architecture}
\label{sec:architecture}

The network that we propose is a one-dimensional (1D) CDAE ...


%%----------------------------------------------------------------------
\subsection{Data Preprocessing and Training}
\label{sec:data}

To better train the CDAE, the input data should be appropriately
preprocessed.


%%----------------------------------------------------------------------
\subsection{Evaluation Index}
\label{sec:index}

To evaluate the EoR signal separation performance, an index must be defined...


%%======================================================================
\section{Experiments}
\label{sec:expriments}

We simulate a data set to train the CDAE and evaluate the performance.
We also carry out comparison of performance between our proposed method
and the traditional polynomial fitting method.


%%----------------------------------------------------------------------
\subsection{Data Simulation}
\label{sec:simulation}

We take the \SIrange{154}{162}{\MHz} frequency band as an example.
We simulate the sky images of the EoR signal, Galactic diffuse emission,
extragalactic point sources, and radio haloes.
The SKA1-Low layout configuration is employed to simulate the SKA1
`observed' images.
In this way, we take into account the complicated instrumental effects
of radio interferometers.
See \citealt{li2018} for more details.


%%----------------------------------------------------------------------
\subsection{Results}
\label{sec:results}

The training results ...
The EoR separation performance ...


%%----------------------------------------------------------------------
\subsection{Comparison}
\label{sec:comparison}

Compare to the traditional polynomial fitting method ...


%%======================================================================
\section{Conclusions}
\label{sec:conclusions}

We have proposed a CDAE to separate the faint EoR signal along the
frequency dimension and achieved excellent results...


%%======================================================================
\section*{Acknowledgements}

This work is supported by
the Ministry of Science and Technology of China
(grant No. 2018YFA0404601),
the National Natural Science Foundation of China
(grant Nos. 11433002, 11621303, 61371147),
and the National Key Research and Discovery Plan
(grant No. 2017YFF0210903).


%%======================================================================
%% References

\bibliographystyle{mnras}
\bibliography{references}


%%======================================================================
%% Appendix

% \appendix


%%======================================================================
% Don't change these lines
\bsp	% typesetting comment
\label{lastpage}
\end{document}

%% EOF
